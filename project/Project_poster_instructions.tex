% Options for packages loaded elsewhere
\PassOptionsToPackage{unicode}{hyperref}
\PassOptionsToPackage{hyphens}{url}
\PassOptionsToPackage{dvipsnames,svgnames,x11names}{xcolor}
%
\documentclass[
  letterpaper,
  DIV=11,
  numbers=noendperiod]{scrartcl}

\usepackage{amsmath,amssymb}
\usepackage{iftex}
\ifPDFTeX
  \usepackage[T1]{fontenc}
  \usepackage[utf8]{inputenc}
  \usepackage{textcomp} % provide euro and other symbols
\else % if luatex or xetex
  \usepackage{unicode-math}
  \defaultfontfeatures{Scale=MatchLowercase}
  \defaultfontfeatures[\rmfamily]{Ligatures=TeX,Scale=1}
\fi
\usepackage{lmodern}
\ifPDFTeX\else  
    % xetex/luatex font selection
\fi
% Use upquote if available, for straight quotes in verbatim environments
\IfFileExists{upquote.sty}{\usepackage{upquote}}{}
\IfFileExists{microtype.sty}{% use microtype if available
  \usepackage[]{microtype}
  \UseMicrotypeSet[protrusion]{basicmath} % disable protrusion for tt fonts
}{}
\makeatletter
\@ifundefined{KOMAClassName}{% if non-KOMA class
  \IfFileExists{parskip.sty}{%
    \usepackage{parskip}
  }{% else
    \setlength{\parindent}{0pt}
    \setlength{\parskip}{6pt plus 2pt minus 1pt}}
}{% if KOMA class
  \KOMAoptions{parskip=half}}
\makeatother
\usepackage{xcolor}
\setlength{\emergencystretch}{3em} % prevent overfull lines
\setcounter{secnumdepth}{-\maxdimen} % remove section numbering
% Make \paragraph and \subparagraph free-standing
\makeatletter
\ifx\paragraph\undefined\else
  \let\oldparagraph\paragraph
  \renewcommand{\paragraph}{
    \@ifstar
      \xxxParagraphStar
      \xxxParagraphNoStar
  }
  \newcommand{\xxxParagraphStar}[1]{\oldparagraph*{#1}\mbox{}}
  \newcommand{\xxxParagraphNoStar}[1]{\oldparagraph{#1}\mbox{}}
\fi
\ifx\subparagraph\undefined\else
  \let\oldsubparagraph\subparagraph
  \renewcommand{\subparagraph}{
    \@ifstar
      \xxxSubParagraphStar
      \xxxSubParagraphNoStar
  }
  \newcommand{\xxxSubParagraphStar}[1]{\oldsubparagraph*{#1}\mbox{}}
  \newcommand{\xxxSubParagraphNoStar}[1]{\oldsubparagraph{#1}\mbox{}}
\fi
\makeatother


\providecommand{\tightlist}{%
  \setlength{\itemsep}{0pt}\setlength{\parskip}{0pt}}\usepackage{longtable,booktabs,array}
\usepackage{calc} % for calculating minipage widths
% Correct order of tables after \paragraph or \subparagraph
\usepackage{etoolbox}
\makeatletter
\patchcmd\longtable{\par}{\if@noskipsec\mbox{}\fi\par}{}{}
\makeatother
% Allow footnotes in longtable head/foot
\IfFileExists{footnotehyper.sty}{\usepackage{footnotehyper}}{\usepackage{footnote}}
\makesavenoteenv{longtable}
\usepackage{graphicx}
\makeatletter
\newsavebox\pandoc@box
\newcommand*\pandocbounded[1]{% scales image to fit in text height/width
  \sbox\pandoc@box{#1}%
  \Gscale@div\@tempa{\textheight}{\dimexpr\ht\pandoc@box+\dp\pandoc@box\relax}%
  \Gscale@div\@tempb{\linewidth}{\wd\pandoc@box}%
  \ifdim\@tempb\p@<\@tempa\p@\let\@tempa\@tempb\fi% select the smaller of both
  \ifdim\@tempa\p@<\p@\scalebox{\@tempa}{\usebox\pandoc@box}%
  \else\usebox{\pandoc@box}%
  \fi%
}
% Set default figure placement to htbp
\def\fps@figure{htbp}
\makeatother

\KOMAoption{captions}{tableheading}
\makeatletter
\@ifpackageloaded{tcolorbox}{}{\usepackage[skins,breakable]{tcolorbox}}
\@ifpackageloaded{fontawesome5}{}{\usepackage{fontawesome5}}
\definecolor{quarto-callout-color}{HTML}{909090}
\definecolor{quarto-callout-note-color}{HTML}{0758E5}
\definecolor{quarto-callout-important-color}{HTML}{CC1914}
\definecolor{quarto-callout-warning-color}{HTML}{EB9113}
\definecolor{quarto-callout-tip-color}{HTML}{00A047}
\definecolor{quarto-callout-caution-color}{HTML}{FC5300}
\definecolor{quarto-callout-color-frame}{HTML}{acacac}
\definecolor{quarto-callout-note-color-frame}{HTML}{4582ec}
\definecolor{quarto-callout-important-color-frame}{HTML}{d9534f}
\definecolor{quarto-callout-warning-color-frame}{HTML}{f0ad4e}
\definecolor{quarto-callout-tip-color-frame}{HTML}{02b875}
\definecolor{quarto-callout-caution-color-frame}{HTML}{fd7e14}
\makeatother
\makeatletter
\@ifpackageloaded{caption}{}{\usepackage{caption}}
\AtBeginDocument{%
\ifdefined\contentsname
  \renewcommand*\contentsname{Table of contents}
\else
  \newcommand\contentsname{Table of contents}
\fi
\ifdefined\listfigurename
  \renewcommand*\listfigurename{List of Figures}
\else
  \newcommand\listfigurename{List of Figures}
\fi
\ifdefined\listtablename
  \renewcommand*\listtablename{List of Tables}
\else
  \newcommand\listtablename{List of Tables}
\fi
\ifdefined\figurename
  \renewcommand*\figurename{Figure}
\else
  \newcommand\figurename{Figure}
\fi
\ifdefined\tablename
  \renewcommand*\tablename{Table}
\else
  \newcommand\tablename{Table}
\fi
}
\@ifpackageloaded{float}{}{\usepackage{float}}
\floatstyle{ruled}
\@ifundefined{c@chapter}{\newfloat{codelisting}{h}{lop}}{\newfloat{codelisting}{h}{lop}[chapter]}
\floatname{codelisting}{Listing}
\newcommand*\listoflistings{\listof{codelisting}{List of Listings}}
\makeatother
\makeatletter
\makeatother
\makeatletter
\@ifpackageloaded{caption}{}{\usepackage{caption}}
\@ifpackageloaded{subcaption}{}{\usepackage{subcaption}}
\makeatother

\usepackage{bookmark}

\IfFileExists{xurl.sty}{\usepackage{xurl}}{} % add URL line breaks if available
\urlstyle{same} % disable monospaced font for URLs
\hypersetup{
  pdftitle={Project Poster Instructions},
  colorlinks=true,
  linkcolor={blue},
  filecolor={Maroon},
  citecolor={Blue},
  urlcolor={Blue},
  pdfcreator={LaTeX via pandoc}}


\title{Project Poster Instructions}
\usepackage{etoolbox}
\makeatletter
\providecommand{\subtitle}[1]{% add subtitle to \maketitle
  \apptocmd{\@title}{\par {\large #1 \par}}{}{}
}
\makeatother
\subtitle{BSTA 512/612}
\author{}
\date{}

\begin{document}
\maketitle


\begin{tcolorbox}[enhanced jigsaw, opacityback=0, opacitybacktitle=0.6, rightrule=.15mm, toprule=.15mm, left=2mm, toptitle=1mm, breakable, colback=white, bottomtitle=1mm, arc=.35mm, leftrule=.75mm, titlerule=0mm, title=\textcolor{quarto-callout-important-color}{\faExclamation}\hspace{0.5em}{Important}, bottomrule=.15mm, coltitle=black, colbacktitle=quarto-callout-important-color!10!white, colframe=quarto-callout-important-color-frame]

Poster instructions will be edited for clarity, specifics about our
dataset, and an in-depth rubric. - Nicky (3/31)

\end{tcolorbox}

\begin{center}\rule{0.5\linewidth}{0.5pt}\end{center}

The purpose section was partially developed using ChatGPT by feeding in
my previous project report instructions and asking ChatGPT to edit for a
poster.

\subsection{Directions}\label{directions}

\subsubsection{Purpose}\label{purpose}

A scientific poster serves as a visual and concise way to communicate
research findings. For this project, your poster should highlight your
linear regression analysis and results while ensuring the context and
methods are clearly explained. Posters should balance visuals (e.g.,
tables, figures) with text to engage an audience effectively.

\subsubsection{Formatting guide}\label{formatting-guide}

\paragraph{Poster specifications}\label{poster-specifications}

\begin{itemize}
\tightlist
\item
  This poster can be done in any program you would like

  \begin{itemize}
  \tightlist
  \item
    Powerpoint is a common way to make a poster

    \begin{itemize}
    \tightlist
    \item
      This option is easier to start but more annoying when you have to
      edit visuals and fix the poster
    \item
      \href{https://www.slidesai.io/blog/how-to-make-a-poster-powerpoint}{Some
      help creating it}
    \end{itemize}
  \item
    There are poster templates in Quarto

    \begin{itemize}
    \tightlist
    \item
      This will be a little harder to start, but easier to edit visuals
    \item
      \href{https://github.com/quarto-ext/typst-templates/tree/main/poster}{Quarto
      template}
    \end{itemize}
  \end{itemize}
\item
  \textbf{Please submit a PDF of your poster}
\item
  Poster dimensions: 36'' by 24''

  \begin{itemize}
  \tightlist
  \item
    Mostly important to keep the ratio as 6:4
  \item
    Use a landscape layout
  \end{itemize}
\item
  Font size should be no less than 36pt
\item
  Sectioning of the report

  \begin{itemize}
  \tightlist
  \item
    Main sections that were required: Introduction, Methods, Results,
    Conclusion, and References
  \item
    Other sections that might help group specific methods or results
  \end{itemize}
\item
  Title information at the top of the poster

  \begin{itemize}
  \tightlist
  \item
    This includes the title itself, your name, and the date
  \end{itemize}
\item
  Poster printing for class on 3/17

  \begin{itemize}
  \tightlist
  \item
    Print in color!!
  \item
    Using your PDF, and opening in Adobe Reader, then use the print
    function
  \item
    You should see something like this:
    \pandocbounded{\includegraphics[keepaspectratio]{../labs/images/poster_printing.png}}
  \item
    Choose the poster print option then choose the appropriate Tile
    Scale to make the poster span 4 pages
  \item
    Note: not all printers accommodate this
  \item
    You should be able to print in Vanport

    \begin{itemize}
    \tightlist
    \item
      I do not want you to pay for printing - please let me know if you
      are unable to print for free
    \item
      Here are two documents in the Student files to help with printing

      \begin{itemize}
      \tightlist
      \item
        \href{https://ohsuitg-my.sharepoint.com/:b:/r/personal/wakim_ohsu_edu/Documents/Teaching/Classes/W25_BSTA_512_612/Student_files/Project/Resources/Printers.pdf?csf=1&web=1&e=2CD9k5}{Printer
        info}
      \item
        \href{https://ohsuitg-my.sharepoint.com/:b:/r/personal/wakim_ohsu_edu/Documents/Teaching/Classes/W25_BSTA_512_612/Student_files/Project/Resources/Printing,\%20Copying,\%20Scanning\%20Overview\%20(OHSU\%20and\%20PSU).pdf?csf=1&web=1&e=K93grX}{Printer
        guidance}
      \end{itemize}
    \end{itemize}
  \item
    \textbf{Posters presented on 3/17 in class do NOT need to be what
    you turn in at 11pm in Sakai}
  \end{itemize}
\end{itemize}

\paragraph{Tables and figures}\label{tables-and-figures}

\begin{itemize}
\tightlist
\item
  Tables and figures should NOT have variable names as they appear in
  the data frame

  \begin{itemize}
  \tightlist
  \item
    Variable names should be understood by a reader
  \item
    Variable names should be written in full words
  \item
    Include a title or caption for all figures
  \item
    Figure and tables appear on same page or close to same page where
    they are first referenced
  \item
    Tables and figures are an appropriate size in the html
  \item
    Nicky is able to read all words in figures and tables
  \end{itemize}
\item
  Figures and tables should be clear and crisp

  \begin{itemize}
  \tightlist
  \item
    Make sure they are not blurred
  \item
    Screenshots are okay, but will likely make them blurry if you're not
    careful
  \item
    Best option is to use the \texttt{save()} to save a jpg or png
  \end{itemize}
\end{itemize}

\paragraph{Writing}\label{writing}

\begin{itemize}
\tightlist
\item
  Writing, spelling, and grammar should be admissible

  \begin{itemize}
  \tightlist
  \item
    This means I can generally follow your thought/what you are trying
    to communicate
  \item
    Some spelling and grammar mistakes are allowed

    \begin{itemize}
    \tightlist
    \item
      I will not take off points if there are a few sprinkled in
    \item
      If \emph{every or close to every} sentence has mistakes, then I
      will take off
    \end{itemize}
  \end{itemize}
\end{itemize}

\begin{tcolorbox}[enhanced jigsaw, opacityback=0, opacitybacktitle=0.6, rightrule=.15mm, toprule=.15mm, left=2mm, toptitle=1mm, breakable, colback=white, bottomtitle=1mm, arc=.35mm, leftrule=.75mm, titlerule=0mm, title=\textcolor{quarto-callout-note-color}{\faInfo}\hspace{0.5em}{The project report is a separate file from the labs}, bottomrule=.15mm, coltitle=black, colbacktitle=quarto-callout-note-color!10!white, colframe=quarto-callout-note-color-frame]

You can save tables and figures from labs or separate files, then load
them in the report

\begin{itemize}
\tightlist
\item
  Save R objects in analyses file:

  \begin{itemize}
  \tightlist
  \item
    Suppose you named the Table 1 as \texttt{table1}
  \item
    \texttt{save(table1,\ file\ =\ "table1.Rdata")}
  \end{itemize}
\item
  Load R objects in report file: \texttt{load(file\ =\ "table1.Rdata")}
\end{itemize}

\end{tcolorbox}

\subsubsection{Poster template}\label{poster-template}

\begin{itemize}
\tightlist
\item
  \href{https://ohsuitg-my.sharepoint.com/:p:/r/personal/wakim_ohsu_edu/Documents/Teaching/Classes/W25_BSTA_512_612/Student_files/Project/Resources/poster_template.pptx?d=w171b4f2e4f1d4ee1a72f499f15d736af&csf=1&web=1&e=Nsqqqs}{Powerpoint
  Template}

  \begin{itemize}
  \tightlist
  \item
    Feel free to adjust this poster visually
  \end{itemize}
\item
  \href{../labs/Poster-example.qmd}{Quarto Template}

  \begin{itemize}
  \tightlist
  \item
    Only works if you followed the steps in the Quarto template in the
    Formatting Guide
  \item
    Feel free to adjust this poster visually
  \end{itemize}
\end{itemize}

\subsubsection{Poster examples}\label{poster-examples}

\begin{itemize}
\tightlist
\item
  \href{https://ohsuitg-my.sharepoint.com/:b:/r/personal/wakim_ohsu_edu/Documents/Teaching/Classes/W25_BSTA_512_612/Student_files/Project/Sample_poster/McVeety_Trans_Single_Payer_Poster.pdf?csf=1&web=1&e=PsN3TI}{Poster
  1}

  \begin{itemize}
  \tightlist
  \item
    Good example of layouts and well-executed visuals
  \end{itemize}
\item
  \href{https://ohsuitg-my.sharepoint.com/:b:/r/personal/wakim_ohsu_edu/Documents/Teaching/Classes/W25_BSTA_512_612/Student_files/Project/Sample_poster/Analysis\%20of\%20Veterans\%27\%20Non-disclosure\%20of\%20Suicidal\%20Thoughts.pdf?csf=1&web=1&e=z7V8cv}{Poster
  2}

  \begin{itemize}
  \tightlist
  \item
    Good example of a forest plot to display coefficient estimates of
    many covariates
  \item
    Good example of patient information displayed
  \item
    Good highlight of the study goals in the background
  \end{itemize}
\item
  \href{https://ohsuitg-my.sharepoint.com/:b:/r/personal/wakim_ohsu_edu/Documents/Teaching/Classes/W25_BSTA_512_612/Student_files/Project/Sample_poster/nwakim_prop_score.pdf?csf=1&web=1&e=xefr7T}{Poster
  3}

  \begin{itemize}
  \tightlist
  \item
    Done by Nicky (a long time ago) so please excuse a lot of the poor
    language around race/ethnicity

    \begin{itemize}
    \tightlist
    \item
      I have learned a lot since then! My statistics education did not
      do a great job of incorporating responsible practice around
      participant's identities
    \end{itemize}
  \item
    Showing this poster because it is a good example of the level of
    detail expected in our poster's background, methods, and conclusions
  \end{itemize}
\end{itemize}

\subsubsection{Grading}\label{grading}

\paragraph{Grading}

The project report is out of 36 points. Note that the Statistical
Methods and Results sections are graded on an 8-point scale, while all
other components are graded on a 4-point scale.

\paragraph{Rubric}

\begin{longtable}[]{@{}
  >{\raggedright\arraybackslash}p{(\linewidth - 10\tabcolsep) * \real{0.1667}}
  >{\raggedright\arraybackslash}p{(\linewidth - 10\tabcolsep) * \real{0.1667}}
  >{\raggedright\arraybackslash}p{(\linewidth - 10\tabcolsep) * \real{0.1667}}
  >{\raggedright\arraybackslash}p{(\linewidth - 10\tabcolsep) * \real{0.1667}}
  >{\raggedright\arraybackslash}p{(\linewidth - 10\tabcolsep) * \real{0.1667}}
  >{\raggedright\arraybackslash}p{(\linewidth - 10\tabcolsep) * \real{0.1667}}@{}}
\toprule\noalign{}
\begin{minipage}[b]{\linewidth}\raggedright
\end{minipage} & \begin{minipage}[b]{\linewidth}\raggedright
4 points
\end{minipage} & \begin{minipage}[b]{\linewidth}\raggedright
3 points
\end{minipage} & \begin{minipage}[b]{\linewidth}\raggedright
2 points
\end{minipage} & \begin{minipage}[b]{\linewidth}\raggedright
1 point
\end{minipage} & \begin{minipage}[b]{\linewidth}\raggedright
0 points
\end{minipage} \\
\midrule\noalign{}
\endhead
\bottomrule\noalign{}
\endlastfoot
Formatting & Poster submitted on Sakai with PDF file. Report is written
in well-constructed bullets with very few grammatical or spelling
errors. With little editing, the poster can be presented at a
conference. & Poster submitted on Sakai with PDF file. Report is written
in well-constructed bullets with some (around 2 per section) grammatical
or spelling errors. With some editing, the report can be distributed. &
Poster submitted on Sakai with PDF file. Report is written in
well-constructed bullets, but have many grammatical or spelling errors.
With major editing, the report can be distributed. & Poster submitted on
Sakai with PDF file. Report is written in well-constructed bullets, but
are very hard to follow due to grammar mistakes. & Poster not submitted
on Sakai. Poster not in PDF file type. Report language cannot be
followed. With major editing, the report can be distributed. \\
Figures and work & All requested output is displayed, including 2
required figures and tables, and at least one additional figure. Figures
and tables look professional, are easily interpreted by the reader, and
easily convey the intended message. & All requested output is displayed,
including 2 required figures and tables, and at least one additional
figure. For the most part, figures and tables look professional, are
easily interpreted by the reader, and easily convey the intended
message. A few mistakes in the figures are made. & All requested output
is displayed, including 2 required figures and tables, and at least one
additional figure. Figures and tables look semi-professional, are not so
easily interpreted by the reader, and convey the intended message but
after some work by the reader. Some mistakes in the figures are made. &
All requested output is displayed, including 2 required figures and
tables, and at least one additional figure. Figures and tables do not
look professional, are not easily interpreted by the reader, and/or do
not convey the intended message. Many mistakes in the figures are made.
& Requested output is not displayed, Missing one or more figures. \\
Introduction & Provides a good background for the research question,
includes motivation for the question, and references previous research
that justifies this analysis. & Provides a decent background for the
research question and includes motivation for the question. Previous
research is mentioned, but feels disconnected to the current analysis. &
Provides a decent background for the research question and includes
motivation for the question. Previous research is mentioned, but feels
disconnected to the current analysis. & Does not provide a background
that connects to the research question. Motivation and previous research
are not mentioned. & No introduction included. \\
Methods (8 points) & Describes statistical methods concisely and
highlights pertinent information to the reader (listed Sections below).
Demonstrates proper analyses were performed. & Describes statistical
methods and highlights pertinent information to the reader (listed
Sections below). Details were omitted or added that were not needed to
explain the overarching methods. Demonstrates proper analyses were
performed. & Describes statistical methods and highlights pertinent
information to the reader (listed Sections below). Details were omitted
or added that were not needed to explain the overarching methods. Some
incorrect analyses included in the description. & Describes statistical
methods, but lacks clarity. Demonstrates a lack of understanding about
the overall process of regression analysis. Incorrect analyses included
in the description. & No methods included. \\
Results (8 points) & Correctly interprets coefficients for the
explanatory variable and identifies any other interesting trends.
Highlights pertinent results to the reader (listed Sections below). &
Correctly interprets coefficients, but does correctly incorporate the
interaction (if in the model). Highlights pertinent results to the
reader (listed Sections below). & Incorrectly interprets coefficients.
Highlights pertinent results to the reader (listed Sections below). &
Incorrectly interprets coefficients.Omits pertinent results to the
reader (listed Sections below). & No results included. \\
Conclusion/ Discussion & Main research question is answered and
statistical caveats described to non-technical person. Thoroughly and
concisely discusses limitations and considerations of the results, and
their consequences. & Main research question is answered and statistical
caveats described to non-technical person. Discusses limitations and
considerations of the results and their consequences, but misses some
big considerations. & Main research question is somewhat answered (but
focus is not on the research question) and statistical caveats described
to non-technical person. Discusses limitations and considerations of the
results, but does not discuss the consequences. & Main research question
is somewhat answered (but not the focus at all) and statistical caveats
are not described.Discusses limitations and considerations of the
results, but misses many considerations and does not discuss
consequences. & No conclusion included. \\
References & References are mostly cited consistently within the report,
and in the Reference section. This includes the data source! &
References are sometimes cited consistently within the report, and in
the Reference section. This includes the data source! & References are
sometimes cited consistently within the report, and in the Reference
section. This includes the data source! & References are not cited
consistently within the report, and in the Reference section. This
includes the data source! & References are not included at all. \\
\end{longtable}

\begin{itemize}
\item
  In formatting, an example of a report with little editing needed is
  one that has zero to some grammar or spelling mistakes, no code chunks
  showing, and no output warnings nor messages showing.
\item
  Professional figures mean

  \begin{itemize}
  \item
    I can read the words and numbers in the html

    \begin{itemize}
    \item
      Variable names are converted from the data frame version to
      readable text
    \item
      For example: \texttt{iam\_001} does not show up on axes, instead
      something like: \texttt{Response\ to\ "Currently,\ I\ am..."}
    \end{itemize}
  \item
    Colors are only used if conveying information
  \item
    Intended message of the figure is easily understood

    \begin{itemize}
    \tightlist
    \item
      If you are trying to show a trend of mean IAT vs.~an ordered
      categorical variable, then the variable is ordered on the x-axis
    \end{itemize}
  \end{itemize}
\item
  For the references

  \begin{itemize}
  \item
    I will not be overly critical about the formatting
  \item
    By consistency, I mean that you if you are citing things like (Last
    Name, Year) it doesn't suddenly change to number citations.
  \item
    If you would like to use Quarto's citation tool, you can! I actually
    pair it with Zotero and it works beautifully! (But I would not
    embark on this if you haven't used Zotero before)
  \end{itemize}
\end{itemize}

\subsection{Sections}\label{sections}

\subsubsection{Title}\label{title}

\begin{itemize}
\tightlist
\item
  \textbf{Purpose:} Create an identifiable name for your research
  project that includes the main research question's variables and gives
  some context to the analysis or results
\end{itemize}

\subsubsection{Introduction}\label{introduction}

\begin{itemize}
\tightlist
\item
  \textbf{Length: 5-8 bullets}
\item
  \textbf{Purpose:} Introduce the research question and why it is
  important to study
\item
  This section is non-technical.

  \begin{itemize}
  \tightlist
  \item
    By reading just the introduction and conclusion, someone without a
    technical background should have an idea of what they study was
    about, why it is important, and what the main results are
  \end{itemize}
\item
  You may start with your bullets from Lab 1, but you should edit it and
  make sure it flows into your report well!
\item
  Should contain some references
\end{itemize}

\subsubsection{Methods}\label{methods}

\begin{itemize}
\tightlist
\item
  \textbf{Length: 8-10 bullets}
\item
  \textbf{Purpose:} Describe the analyses that were conducted and
  methods used to select variables and check diagnostics
\item
  \textbf{Some important methods to discuss} (You may divide these into
  your sections, not necessarily with these names)

  \begin{itemize}
  \tightlist
  \item
    General approach to the dataset

    \begin{itemize}
    \tightlist
    \item
      2-3 bullets
    \item
      Where did the data come from?
    \item
      Did you need to do any quality control?
    \item
      Missing data: we performed complete case analysis

      \begin{itemize}
      \tightlist
      \item
        1 bullet
      \item
        Can be included in the Exploratory data analysis section
      \end{itemize}
    \item
      What program did you use to analyze the data?
    \end{itemize}
  \item
    Variables and variable creation

    \begin{itemize}
    \tightlist
    \item
      This includes a description of analyses for Table 1 and what
      statistics were used to summarize the variables

      \begin{itemize}
      \tightlist
      \item
        More on creation of Table 1, not discussing the results of Table
        1
      \end{itemize}
    \item
      Includes (only include if you did one of the following)

      \begin{itemize}
      \tightlist
      \item
        Indicators for gender identity or race
      \item
        Creating BMI
      \item
        Categorizing a continuous variable (even if performed in model
        selection)
      \item
        Using scoring for an ordered categorical variable (that is not
        your explanatory variable)
      \end{itemize}
    \item
      1 bullet for all variables
    \end{itemize}
  \item
    Model building: we performed purposeful selection

    \begin{itemize}
    \tightlist
    \item
      1-3 bullets
    \item
      Includes

      \begin{itemize}
      \tightlist
      \item
        Describe purposeful selection: combining existing literature,
        clinical significance, and analysis
      \item
        How did you build the model? Describe the process
      \item
        Did you consider confounders and effect modifiers?
      \end{itemize}
    \item
      Example: We considered the following potential confounders: list
      fo them. Based on our research question, existing literature, and
      clinical significance, we used purposeful selection to identify
      confounders and effect modifiers.
    \end{itemize}
  \item
    Final model

    \begin{itemize}
    \tightlist
    \item
      1 bullet
    \item
      Write out your final model equation
    \item
      I would include the main exploratory variable then a placeholder
      for all the other covariates
    \end{itemize}
  \item
    Model diagnostics

    \begin{itemize}
    \tightlist
    \item
      1-3 sentences
    \item
      Includes

      \begin{itemize}
      \tightlist
      \item
        Process of investigating model diagnostics
      \item
        By the time you build the model, LINE assumptions should be met
      \item
        If assumptions were not met, what process did you use to fix it?
      \end{itemize}
    \end{itemize}
  \end{itemize}
\end{itemize}

\begin{tcolorbox}[enhanced jigsaw, opacityback=0, opacitybacktitle=0.6, rightrule=.15mm, toprule=.15mm, left=2mm, toptitle=1mm, breakable, colback=white, bottomtitle=1mm, arc=.35mm, leftrule=.75mm, titlerule=0mm, title=\textcolor{quarto-callout-caution-color}{\faFire}\hspace{0.5em}{Important to keep in mind}, bottomrule=.15mm, coltitle=black, colbacktitle=quarto-callout-caution-color!10!white, colframe=quarto-callout-caution-color-frame]

Methods typically describe your approach and process, not the results of
that process

\begin{itemize}
\tightlist
\item
  For example: I might say ``We investigated the linearity of each
  continuous covariate visually. If continuous variables were not
  linear, then we divided the variable into categories using existing
  guidelines from \textless insert reference here\textgreater{} or
  creating quartiles.''
\item
  In the methods section, I would NOT say: ``We investigated the
  linearity of each continuous covariate visually. We found that age was
  not linearly related to IAT scores. Thus, we categorized age into the
  following groups: \_\_\_, \_\_\_\_, \_\_\_\_, \_\_\_\_, and
  \_\_\_\_.''
\item
  The last two sentences about age would be more appropriate in the
  Results section
\end{itemize}

\end{tcolorbox}

\subsubsection{Results}\label{results}

\begin{tcolorbox}[enhanced jigsaw, opacityback=0, opacitybacktitle=0.6, rightrule=.15mm, toprule=.15mm, left=2mm, toptitle=1mm, breakable, colback=white, bottomtitle=1mm, arc=.35mm, leftrule=.75mm, titlerule=0mm, title=\textcolor{quarto-callout-caution-color}{\faFire}\hspace{0.5em}{Caution}, bottomrule=.15mm, coltitle=black, colbacktitle=quarto-callout-caution-color!10!white, colframe=quarto-callout-caution-color-frame]

Note: I took out ``and \textbf{stratified by your primary independent
variable}'' for Table 1.

\end{tcolorbox}

\begin{itemize}
\tightlist
\item
  \textbf{Length: mostly figures with 2-3 bullet points}
\item
  \textbf{Purpose:} Relay the results from our sample's analysis
  typically focusing on the numbers and interpretations
\item
  Tables \& figures (2-3 tables or figures)

  \begin{itemize}
  \tightlist
  \item
    The following are required tables or figures

    \begin{itemize}
    \tightlist
    \item
      Table 1 summarizing participant characteristics both
      \textbf{overall}
    \item
      Table or figure with regression results

      \begin{itemize}
      \tightlist
      \item
        Can be a forest plot
      \item
        If you have A LOT of coefficient estimates, the forest plot may
        not work well!
      \end{itemize}
    \end{itemize}
  \item
    \textbf{1-2 figures that you think are helpful in understanding the
    results, for example}

    \begin{itemize}
    \tightlist
    \item
      DAG explaining connection between variables (if you did this)
    \item
      Table or figure to compare model fit statistics (if you did this)
    \item
      Table or figure for unadjusted relationship between outcome and
      explanatory variables
    \end{itemize}
  \end{itemize}
\item
  Interpret the \textbf{important} model coefficients in the context of
  the research question

  \begin{itemize}
  \tightlist
  \item
    2-3 bullets
  \item
    Interpreting the explanatory variable's relationship with IAT score
    is the most important thing to report!!

    \begin{itemize}
    \tightlist
    \item
      When doing this, make sure you account for ALL interactions: If
      your explanatory variable has multiple interactions and you are
      trying to interpret one, then what does that mean about the other
      variables involved in the other interactions? If this is
      confusing, please make an appointment with me!!
    \end{itemize}
  \end{itemize}
\end{itemize}

\subsubsection{Conclusion}\label{conclusion}

\begin{itemize}
\tightlist
\item
  \textbf{Length: 3-6 bullets}
\item
  \textbf{Purpose:} Describe the main conclusions to a non-technical
  audience
\item
  What was the answer to your research question?

  \begin{itemize}
  \tightlist
  \item
    Mention the direction of the association if there was one
  \end{itemize}
\item
  Any other interesting results?
\end{itemize}

\subsubsection{Discussion}\label{discussion}

\begin{itemize}
\tightlist
\item
  \textbf{Length: 3-5 bullets}
\item
  \textbf{Purpose:} Discuss the results and give them context outside of
  the sample and its analysis
\item
  Some important things to include

  \begin{itemize}
  \tightlist
  \item
    Include limitations of the results

    \begin{itemize}
    \tightlist
    \item
      You don't need to hit all the limitations, but think about the big
      ones (generalizability? independence of samples? large sample size
      vs.~clinical significance? the way we handled variables?)
    \end{itemize}
  \item
    After limitations, discuss the positive parts of the results

    \begin{itemize}
    \tightlist
    \item
      What can we do with these results? What impact can it have?
    \end{itemize}
  \item
    Any overarching trends that are worth noting? {[}@Giebel2024{]}
  \end{itemize}
\item
  Should contain some references
\end{itemize}

\subsubsection{References}\label{references}

\begin{itemize}
\tightlist
\item
  Include your references here!
\item
  You introduction should have references, especially when discussing
  the social science behind the analysis
\item
  You must reference the IAT data source!!
\end{itemize}




\end{document}
